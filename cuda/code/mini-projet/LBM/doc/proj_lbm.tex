\documentclass[utf8,usenames,dvipsnames,hyperref={pdfpagemode=FullScreen,urlcolor=blue}]{beamer}

\usepackage[frenchb]{babel}
\usepackage[utf8]{inputenc}
\usepackage{fontspec}
%\setmainfont{Arial}
%\setmainfont{Broadway}

\usepackage{commath}
\usepackage{amsmath,amssymb,amsfonts}
%\usepackage{mathspec}
%\setmathsfont(Digits,Latin,Greek)[Numbers={Lining,Proportional}]{Broadway}

\usepackage{pgfpages}

\usepackage{xcolor}
\definecolor{myblue}{RGB}{0,153,255}
\definecolor{myyellow}{RGB}{234,187,0}
\colorlet{darkgreen}{OliveGreen}


%%%%%%%%%%%%%%%%%%%
\hypersetup{%
  pdftitle={KESTENER-MASTER-SPECIALISE-HPC-IA-2021},%
  pdfauthor={Pierre Kestener - CEA Saclay - https://github.com/pkestene},
  pdfsubject={MASTER - HPC/IA - 2021},
  pdfkeywords={CFD, GPU, CUDA, EXASCALE, PERFORMANCE PORTABILITY, KOKKOS, SDM, HIGH-ORDER SCHEMES},
  pdfproducer={pdflatex avec la classe BEAMER},
  bookmarksopen=false,
  urlcolor=blue
%  pdfpagemode=FullScreen
}

\newcommand\myurl[1]{\textcolor{purple}{\underline{\url{#1}}}}
\newcommand\myhref[2]{\textcolor{purple}{\underline{\href{#1}{#2}}}}


\title{Mini-projet du cours de programmation GPU}
\subtitle{}
\date{11 janvier 2021}
\author[P. Kestener]{Pierre KESTENER}
\institute{CEA / DRF / IFRU / DEDIP\\
Master spécialisé HPC-IA, Mines Paritech, Sophia Antipolis}
\titlegraphic{\flushright\includegraphics[height=1.5cm]{graphics/logo_cea_verticale}}

\mode<presentation>
\usetheme{cea}
% {
%   \usetheme{Warsaw}
%   \setbeamertemplate{navigation symbols}{}

%   \usecolortheme{beaver}
%   \usecolortheme{lily,sidebartab}

%   \usefonttheme{serif}

%   \setbeamertemplate{footline}[page number]
%   \setbeamertemplate{sidebar canvas right}[vertical shading][top=palette
%   primary.bg,%,middle=white,
%   bottom=palette primary.bg]

%   \useinnertheme{default}
% }

% uncomment if you do want notes
% \setbeameroption{show notes on second screen=right}

% enlarge
\setbeamersize{text margin left=5pt,text margin right=5pt}

% comment if you do not want grey item
\setbeamercovered{transparent}

% comment if you do not want a summary at each section
\AtBeginSection[]{
  \begin{frame}{Summary}
    \small \tableofcontents[currentsection, hideothersubsections]
  \end{frame}
}


\begin{document}

%%%%%%%%%%%%%%%%%%%%%%%%%%%%%%%%%%%%%%%%%%%%%%%%%%%%%%%%%%%%%%%%%%%%%%%%%%
% PAGE DE TITRE
%%%%%%%%%%%%%%%%%%%%%%%%%%%%%%%%%%%%%%%%%%%%%%%%%%%%%%%%%%%%%%%%%%%%%%%%%%
\begin{frame}
  \titlepage
\end{frame}

%%%%%%%%%%%%%%%%%%%%%%%%%%%%%%%%%%%%%%%%%%%%%%%%%%%%%%%%%%%%%%%%%%%%%%%%%%
% LBM
%%%%%%%%%%%%%%%%%%%%%%%%%%%%%%%%%%%%%%%%%%%%%%%%%%%%%%%%%%%%%%%%%%%%%%%%%%
\begin{frame}
  \frametitle{La méthode dite de Boltzmann sur réseau}
  % \framesubtitle{My subtitle}

  \begin{itemize}
  \item LBM (Lattice Boltzmann Method) : simulation d'écoulement fluide
  \end{itemize}

  \begin{center}
    \includegraphics[width=0.7\linewidth]{./images/lbm_cyl}
  \end{center}

\end{frame}

%%%%%%%%%%%%%%%%%%%%%%%%%%%%%%%%%%%%%%%%%%%%%%%%%%%%%%%%%%%%%%%%%%%%%%%%%%
%%%%%%%%%%%%%%%%%%%%%%%%%%%%%%%%%%%%%%%%%%%%%%%%%%%%%%%%%%%%%%%%%%%%%%%%%%
\begin{frame}
  \frametitle{Théorie cinétique des gaz}
  % \framesubtitle{My subtitle}

  \begin{itemize}
  \item \myhref{https://fr.wikipedia.org/wiki/Th\%C3\%A9orie_cin\%C3\%A9tique_des_gaz}{théorie cinétique des gaz}
  \item fonction de distribution $f(\overrightarrow{\mathbf{r}},\overrightarrow{\mathbf{c}},t)$, avec $\overrightarrow{\mathbf{r}} \in \mathbb{R}^{3}$ et $\overrightarrow{\mathbf{c}} \in \mathbb{R}^{3}$
  \item interprétation de $f(\overrightarrow{\mathbf{r}},\overrightarrow{\mathbf{c}},t) \dif \overrightarrow{\mathbf{r}} \dif \overrightarrow{\mathbf{c}}$ : nombre de particules dans le volume élémentaire $\dif \overrightarrow{\mathbf{r}}$ et de vitesse comprise entre $\overrightarrow{\mathbf{c}}$ et $\overrightarrow{\mathbf{c}} + \dif \overrightarrow{\mathbf{c}}$
  \item les \textcolor{blue}{\bf grandeurs hydrodynamiques macroscopiques usuelles} (densité, vitesse, énergie, ...) sont définies comme les moments de la fonction de distribution:\\
    \begin{itemize}
    \item \textcolor{red}{\bf densité} : $\rho(\overrightarrow{\mathbf{r}},t) = \int f(\overrightarrow{\mathbf{r}},\overrightarrow{\mathbf{c}},t) \dif \overrightarrow{\mathbf{c}} $
    \item \textcolor{red}{\bf quantité de mouvement} : $\overrightarrow{\mathbf{p}}(\overrightarrow{\mathbf{r}},t) = \int f(\overrightarrow{\mathbf{r}},\overrightarrow{\mathbf{c}},t) \overrightarrow{\mathbf{c}} \dif \overrightarrow{\mathbf{c}} = \rho \overrightarrow{\mathbf{v}}$
    \item \textcolor{red}{\bf énergie cinétique} $e_{{cin}}(\overrightarrow{\mathbf{r}},t) = \int f(\overrightarrow{\mathbf{r}},\overrightarrow{\mathbf{c}},t) ||\overrightarrow{\mathbf{c}}||^{2} \dif \overrightarrow{\mathbf{c}} = \rho \overrightarrow{\mathbf{v}}$
    \end{itemize}
  \item si on sait faire évoluer dans le temps $f$, alors on peut connaitre la dynamique du mouvement du fluide $\Rightarrow$ \myhref{https://en.wikipedia.org/wiki/Boltzmann_equation}{équation de boltzmann}
  \end{itemize}

\end{frame}

%%%%%%%%%%%%%%%%%%%%%%%%%%%%%%%%%%%%%%%%%%%%%%%%%%%%%%%%%%%%%%%%%%%%%%%%%%
%%%%%%%%%%%%%%%%%%%%%%%%%%%%%%%%%%%%%%%%%%%%%%%%%%%%%%%%%%%%%%%%%%%%%%%%%%
\begin{frame}
  \frametitle{L'équation de Boltzmann}

  équation de Boltzmann est une équation de transport qui décrit comment évolue dans le temps la distribution des particules

  \begin{equation*}
    \frac{\partial f}{\partial t} + \overrightarrow{\mathbf{c}}\cdot\nabla_{\overrightarrow{\mathbf{r}}} f + \overrightarrow{\mathbf{F}} \cdot \frac{\partial f}{\partial \overrightarrow{\mathbf{c}}} = \left(\frac{\partial f}{\partial t} \right)_\mathrm{coll}
  \end{equation*}

\end{frame}

%%%%%%%%%%%%%%%%%%%%%%%%%%%%%%%%%%%%%%%%%%%%%%%%%%%%%%%%%%%%%%%%%%%%%%%%%%
%%%%%%%%%%%%%%%%%%%%%%%%%%%%%%%%%%%%%%%%%%%%%%%%%%%%%%%%%%%%%%%%%%%%%%%%%%
\begin{frame}
  \frametitle{La méthode dite de Boltzmann sur réseau}

  \begin{itemize}
  \item \textcolor{red}{\bf Problème:} pour simuler numériquement, il faut discrétiser $f(\overrightarrow{\mathbf{r}},\overrightarrow{\mathbf{c}},t)$ sachant que $\overrightarrow{\mathbf{r}} \in \mathbb{R}^{3}$ et $\overrightarrow{\mathbf{c}} \in \mathbb{R}^{3}$ $\Rightarrow$ le nombre de variables est rapidement gigantesque ($\sim N^{6}$)\\
exemple : $N=100 \Rightarrow N^{6}= 10^{12}$ soit 8 TBytes de mémoire vive juste pour stocker $f$ à un pas de temps, c'est impraticable en pratique.
  \item \textcolor{darkgreen}{\bf Solution:} on discrétise l'espace (variable $\overrightarrow{\mathbf{r}}$ sur une grille $N^{3}$ et la vitesse $\overrightarrow{\mathbf{c}}$ sur un petit nombre (quelque unités à quels dizaines); cela s'appelle la méthode dite de Boltzmann sur réseau qui donne des simulations numériques en bon accord avec les observations expérimentales.\\
    avantages principaux des méthodes LBM: facile à mettre en \oe uvre numériquement et facile à paralléliser.
  \end{itemize}

\end{frame}

%%%%%%%%%%%%%%%%%%%%%%%%%%%%%%%%%%%%%%%%%%%%%%%%%%%%%%%%%%%%%%%%%%%%%%%%%%
%%%%%%%%%%%%%%%%%%%%%%%%%%%%%%%%%%%%%%%%%%%%%%%%%%%%%%%%%%%%%%%%%%%%%%%%%%
\begin{frame}
  \frametitle{La méthode dite de Boltzmann sur réseau}

  \begin{minipage}{0.66\linewidth}
    De manière simplifiée, on peut dire que l'évolution temporelle de la fonction de distribution (où plus exactement des $q$ composantes de $f$) est calculée en itérant les deux opérations suivantes:

    \begin{itemize}
    \item collision: $f_i(\vec{x},t+\delta_t) = f_i(\vec{x},t) + \frac{f_i^{eq}(\vec{x},t)-f_i(\vec{x},t)}{\tau_f}$, avec $ 0 \leq i < q$\\
      le terme de collision choisi ici modélise un retour (relaxation) vers la distribution d'équilibre 
    \item stream: $f_i(\vec{x}+\vec{e}_i,t+1) =f_i(\vec{x},t)$\\
      les particules modélisées par $f_{i}$ se déplacent dans la direction $e_{i}$
    \end{itemize}

  \end{minipage}
  %
  \begin{minipage}{0.33\linewidth}
    \includegraphics[width=0.9\linewidth]{./images/1280px-LB_D2Q9arrows.png}
  \end{minipage}

\end{frame}

%%%%%%%%%%%%%%%%%%%%%%%%%%%%%%%%%%%%%%%%%%%%%%%%%%%%%%%%%%%%%%%%%%%%%%%%%%
%%%%%%%%%%%%%%%%%%%%%%%%%%%%%%%%%%%%%%%%%%%%%%%%%%%%%%%%%%%%%%%%%%%%%%%%%%
\begin{frame}
  \frametitle{Mini-projet}

  \textcolor{red}{Objectif:} En se basant soit sur la version python, soit sur la version C++, et en utilisant le cours de programmation des GPUs, proposer une version parallèlisée de la méthode LBM.

  \begin{itemize}
  \item Choisir le modèle de programmation adapté pour la parallélisation:
    \begin{itemize}
    \item numba pour python
    \item OpenACC pour C++
    \item CUDA/C++ pour les plus courageux ;)
    \end{itemize}
  \item Fournir un mini rapport détaillant votre démarche (identification des noyaux de calcul) et une étude de performance: on pourra par exemple présenté une courbe de speedup (rapport du temps de calcul de la version séquentielle fournie sur le temps de calcul de la version parallèle).
  \end{itemize}


\end{frame}

\end{document}

